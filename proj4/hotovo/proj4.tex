\documentclass[a4paper,11pt,times]{article}

\usepackage[czech]{babel}
\usepackage[utf8]{inputenc}


\usepackage[numbers]{natbib}
\usepackage{graphicx}
\usepackage{ragged2e}
\usepackage{geometry}
\usepackage{setspace}
\usepackage{url}
\DeclareUrlCommand\url{\def\UrlLeft{<}\def\UrlRight{>} \urlstyle{tt}}
\usepackage{hyperref}
\usepackage{pdflscape}

 \geometry{
 a4paper,
 text={170mm,240mm},
 left=20mm,
 top=30mm,
 }

 %konec hlavicky

\begin{document}

\begin{titlepage}
\begin{justify}
\Huge

\centering{\textsc{{{Vysoké učení technické v Brně \\ \huge{Fakulta informačních technologií}\\}}}}
\end{justify}
\vspace{\stretch{0.382}}
\begin{center}
\large
\textbf{\setstretch{0.3}{
Typografie a publikování – 4. projekt\\[0.4em]
\LARGE
Citace\\
\vspace{\stretch{0.618}}}}
\end{center}

\Large
{\LARGE \today \hfill Jan Zbořil (xzbori20)}
\end{titlepage}

\section*{O \LaTeX u}
\LaTeX~je software pro sazbu dokumentů. Není to program typu Microsoft Word, podobá se spíše tvorbě webových stránek pomocí HTML, viz \cite{KottwitzLatexBegginersGuide}. \LaTeX~je dostupný zdarma jako open source software. Skládá se z velkého množství značkovacích příkazů, viz \cite{KopkaGuideToLatex}.

Existuje mnoho způsobů, jak se slovo \LaTeX~vyslovuje, mezi ně patří například [leitek], [la:tek], [lateks] nebo ['leiteks]. Správná česká výslovnost je však [latech], dostupné na \cite{wiki:Latex}.

Tvorba dokumenů v systému \LaTeX~ není triviální, začátečník však může využít mnoha internetových zdrojů, jež ho naučí, jak napříkald správně sázet v dokumentu obrázky, viz \cite{MartinekObrazky}, nebo jak vyzrát na zrádnou podporu českého jazyka, dostupné na \cite{MartinekCestina}.

\LaTeX~je využíván především pro psaní vědeckých prací, například tento článek \cite{LatexClanek} byl vytvořen pomocí \LaTeX u, či vysokoškolských závěrečných prací. Za příklad uvádím dvě práce z FIT VUT: \cite{FIT1} a \cite{FIT2}.

Na FIT VUT se od roku 2005, dle \cite{FITPUB8568}, vyučuje předmět \uv{Typografie a publikování}, který seznamuje studenty s prací v \LaTeX u. Historie odvětví typografie je však daleko delší, příkladem může být například  časopis \uv{Typografia} vycházející od roku 1888 \cite{Typografia}.

\newpage

\bibliographystyle{czechiso}
\bibliography{references}
\end{document}
