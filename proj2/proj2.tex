\documentclass[a4paper,11pt,times,twocolumn]{article}

\usepackage[czech]{babel}
\usepackage[IL2]{fontenc}
\usepackage[utf8]{inputenc}

\usepackage{natbib}
\usepackage{graphicx}
\usepackage{ragged2e}
\usepackage{multicol}
\usepackage{geometry}
\usepackage{amssymb}
\usepackage{setspace}
\usepackage{stackrel}
\usepackage{times}
\usepackage{amsthm}

\newcommand{\N}{\mathbb{N}}

 \geometry{
 a4paper,
 text={180mm,250mm},
 left=15mm,
 top=25mm,
 }
 
\newtheorem{theorem}{Definice}
\newtheorem{veta}{Věta}
 
 %konec hlavicky

\begin{document}


\begin{titlepage}
\begin{justify}
\Huge

\centering{\textsc{{{Fakulta informačních technologií\\ Vysoké učení technické v Brně\\}}}}
\end{justify}
\vspace{\stretch{0.382}}
\begin{center}
\LARGE
\textbf{\setstretch{0.3}{
Typografie a publikování – 2. projekt\\[0.4em]
Sazba dokumentů a matematických výrazů\\
\vspace{\stretch{0.618}}}}
\end{center}

\Large
{\LARGE \today \hfill Jan Zbořil (xzbori20)}
\end{titlepage}

\section*{Úvod}

V této úloze si vyzkoušíme sazbu titulní strany, matematických vzorců, prostředí a dalších textových struktur obvyklých pro technicky zaměřené texty (například rovnice (\ref{eq2}) nebo Definice \ref{Definice 2} na straně 1). Pro vytvoření těchto odkazů používáme příkazy \verb \label , \verb \ref  a \verb \pageref .

\par
\setlength{\parindent}{0.4em}
Na titulní straně je využito sázení nadpisu podle optického středu s využitím zlatého řezu. Tento postup byl
probírán na přednášce. Dále je použito odřádkování se zadanou relativní velikostí 0.4em a 0.3em.

\section{Matematický text}

Nejprve se podíváme na sázení matematických symbolů
a výrazů v plynulém textu včetně sazby definic a vět s využitím balíku \verb amsthm . Rovněž použijeme poznámku pod čarou s použitím příkazu  \verb \footnote . Někdy je vhodné použít konstrukci \verb ${}$ ~nebo \verb \mbox{} , která říká, že (matematický) text nemá být zalomen. V následující definici je nastavena mezera mezi jednotlivými položkami \verb \item  na 0.05em.

\begin{theorem} \label{Definice 1}
\emph{Turingův stroj} (TS) je definován jako šestice tvaru \(M = (Q, \Sigma, \Gamma, \delta, q_0, q_F )\), kde:
\begin{itemize}
    \setlength\itemsep{0.05em}
    \item Q je konečná množina \emph{vnitřních (řídicích) stavů,}
    \item \(\Sigma\) je konečná množina symbolů nazývaná \emph{vstupní abeceda,} \( \Delta \notin \Sigma,\)
    \item $\Gamma$ je konečná množina symbolů, $ \Sigma \subset \Gamma$, $\Delta \in \Gamma$, nazývaná \emph{pásková abeceda,}
    \item $\delta$ : $(Q\backslash \{q_F\} ) \times \Gamma \rightarrow Q \times (\Gamma \cup \{L, R\})$, kde $L, R \notin \Gamma$, je parciální přechodová funkce, a
    \item $q_0 \in Q$ je počáteční stav a $q_f \in Q$ je koncový stav.
\end{itemize}
\end{theorem}

\par
Symbol $\Delta$ značí tzv. \emph{blank} (prázdný symbol), který se
vyskytuje na místech pásky, která nebyla ještě použita.

\emph{Konfigurace pásky} se skládá z nekonečného řetězce, který reprezentuje obsah pásky a pozice hlavy na tomto řetězci. Jedná se o prvek množiny $ \{ \gamma \Delta^\omega ~|~ \gamma \in \Gamma^* \} \times \mathbb{N} \footnote{Pro libovolnou abecedu $\Sigma$ je $\Sigma^\omega$ množina všech \emph{nekonečných} řetězců nad $\Sigma$, tj. nekonečných posloupností symbolů ze $\Sigma$.}.$
\emph{Konfiguraci pásky} obvykle zapisujeme jako $ \Delta xyz \underline{z} x \Delta $ ...
(podtržení značí pozici hlavy). Konfigurace stroje je pak dána stavem řízení a konfigurací pásky. Formálně se jedná o prvek množiny $ \{Q \times \gamma \Delta^\omega ~|~ \gamma \in \Gamma^*  \} \times \mathbb{N} $ .

\subsection{Podsekce obsahující větu a odkaz}
\begin{theorem} \label{Definice 2}
\emph{Řetězec \emph{w} nad abecedou $\Sigma$ je přijat TS} M
jestliže M při aktivaci z počáteční konfigurace pásky $\underline{\Delta} w \Delta...$ a počátečního stavu $q_0$ zastaví přechodem do koncového stavu \mbox{$q_F$, tj. $(q_0, \Delta w \Delta^\omega, 0) \stackrel[M]{*}{\vdash} (q_F , \gamma, n)$} pro
nějaké~\mbox{$ \gamma \in \Gamma^* a~n \in \N $}. 
\par
Množinu $L(M) = \{ w ~ | ~ w $ je přijat $TS~M\} \subseteq \Sigma^*$ nazýváme \emph{jazyk přijímaný TS} M.
\end{theorem}
\par
Nyní si vyzkoušíme sazbu vět a důkazů opět s použitím balíku \verb amsthm .
\begin{veta}
Třída jazyků, které jsou přijímány TS, odpovídá rekurzivně vyčíslitelným jazykům.
\end{veta}

\begin{proof}
V důkaze vyjdeme z Definice \ref{Definice 1} a \ref{Definice 2}.
\end{proof}

\section{Rovnice}
Složitější matematické formulace sázíme mimo plynulý
text. Lze umístit několik výrazů na jeden řádek, ale pak je
třeba tyto vhodně oddělit, například příkazem \verb \quad .

\begin{center}
    $\sqrt[i]{x^3_i}$\quad kde $x_i$ je \textit{i}-té sudé číslo $\quad y_i^{2 \cdot y_i} \ne y_i^{y_i^{y_i}}$
\end{center}

V rovnici (\ref{eq1}) jsou využity tři typy závorek s různou explicitně definovanou velikostí.

\begin{equation} \label{eq1}
x = \bigg\{ \Big( [a+b]*c \Big) ^d \oplus 1 \bigg\}
\end{equation}

\begin{equation} \label{eq2}
y = \lim_{x \to \infty} \frac{\sin^2 x + \cos^2 x}{\frac{1}{\log_{10} x}}
\end{equation}

V této větě vidíme, jak vypadá implicitní vysázení limity $\lim_{n \to \infty} f(n)$ v normálním odstavci textu. Podobně je to i s dalšími symboly jako \mbox{$\sum^n_{i=1} 2^i$  či  $\cap_{A \in \beta } A $}.

V případě vzorců $\lim\limits_{x \to \infty} f(n) $ a $\sum\limits_{i=1}^n 2^i $ jsme si vynutili méně úspornou sazbu příkazem \verb \limits .
\section{Matice}
Pro sázení matic se velmi často používá prostředí \verb array  a závorky (\verb \left , \verb \right ).
\begin{center}
$$
    \left( \begin{array}{ccc}
    a+b  &  \widehat{\xi + \omega}  &  \hat{\pi}\\
    \vec{\textsf{a}}  &  \stackrel{\longleftrightarrow}{AC}  &  \beta\\
    \end{array}
    \right) = 1 \iff \mathbb{Q} = \mathcal{R}
$$
\end{center}

Prostředí \verb array  lze úspěšně využít i jinde.

    $$ \left( \begin{array}{c}
    n\\k 
    \end{array} \right) = \left\{ \begin{array}{cl} 0 & \textrm{pro } k < 0 \textrm{ nebo } k > n \\
    \frac{n!}{k!(n-k)!} & \textrm{pro } 0 \leq k \leq n. \\
    \end{array}\right.
    $$

\end{document}
